\chapter{Campaign C23}\label{sec:c23}

This section provides an overview of the C23 campaign, including some representative graphics of the discharges among time. This section focuses on the signals used by \ac{APODIS} algorithm.

\section{Non-disruptive Discharges}

As \autoref{fig:c23-nondisruptive-1} and \autoref{fig:c23-nondisruptive-2} show, the non-disruptive C23 campaign's discharges are characterized by a stable and predictable pattern. There are some minor differences, for example, total imput power (signal 7) on discharge 74469 is degraded near second 60, or, on discharges 74473 and 74474, plasma density (signal 4) is slightly different.

\begin{figure}[H]
    \centering
    \includegraphics[width=\textwidth]{C23/C23-non-disruptive-1.png}
    \caption{Non-disruptive discharges from the C23 campaign (1)}
    \label{fig:c23-nondisruptive-1}    
\end{figure}

\begin{figure}[H]
    \centering
    \includegraphics[width=\textwidth]{C23/C23-non-disruptive-2.png}
    \caption{Non-disruptive discharges from the C23 campaign (2)}
    \label{fig:c23-nondisruptive-2}    
\end{figure}

\section{Disruptive Discharges}

Disruptive discharges, as shown in \autoref{fig:c23-disruptive}, exhibit significant deviations from the expected patterns. These anomalies can be identified by abrupt changes in the signals. For comparison, the first discharge (74477) is a non-disruptive discharge, while the others are disruptive. Disruptive discharges are usually shorter than non-disruptive ones, as they are stopped to avoid damage to the machine. 

\begin{figure}[H]
    \centering
    \includegraphics[width=\textwidth]{C23/C23-disruptive.png}
    \caption{Disruptive discharges from the C23 campaign}
    \label{fig:c23-disruptive}    
\end{figure}
