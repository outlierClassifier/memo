Las disrupciones en plasmas de tokamak—colapso térmico rápido seguido de extinción de corriente, impulsados por inestabilidades magnetohidrodinámicas (MHD)—imponen riesgos termo-mecánicos y electromagnéticos que escalan de forma desfavorable con el tamaño del dispositivo. 
Anticipar estos eventos con el margen temporal suficiente es condición previa para la protección de la máquina y para habilitar una terminación o mitigación efectiva (p.\,ej., inyección masiva de gas o de \emph{pellets}). 
Este trabajo presenta un ensamble orientado por protocolo para la evaluación de disrupciones que integra modelos heterogéneos tras una interfaz \acs{HTTP}/OpenAPI unificada, habilitando componibilidad, versionado y despliegue entre lenguajes y \emph{runtimes}. 
El modelo de datos refleja el montaje experimental: siete diagnósticos síncronos muestreados cada 1\,ms desde el inicio de la descarga hasta su fin o disrupción. En su mayoría, el desbalance de clases proviene principalmente de la prevalencia de eventos (\(\sim\!6\%\) de descargas son disruptivas); la diferencia adicional de longitud entre descargas no disruptivas (\(\sim\!12\mathrm{k}\) muestras) y disruptivas (\(\sim\!7\mathrm{k}\) muestras) amplifica el desbalance cuando se entrena con ventanas de tamaño fijo. 
El preprocesado es causal en todos los modelos de predicción: las características por ventana y los normalizadores operan sin acceso a muestras futuras. Los descriptores informados por la física—medias por ventana, contenido espectral sin componente continua y cocientes simples que capturan la proximidad a límites operativos—preservan la interpretabilidad a la vez que enfatizan precursores relevantes para evitación y mitigación. 
El ensamble admite tanto evaluación con ventanas fijas como alerta temprana deslizante en flujo, exponiendo puntuaciones calibradas y justificaciones a nivel de ventana para sustentar una lógica de decisión conservadora en el orquestador. 
La generalización se trata como un riesgo de primer orden: la bibliografía reporta la pobre extrapolación de predictores específicos de dispositivo y subraya la necesidad de bases de datos multi-máquina, bien normalizadas y adimensionales para su aplicabilidad en máquinas de siguiente paso. 
En consecuencia, el flujo de trabajo se centra en validación entre campañas, diagnóstico de desplazamiento de covariables y la incorporación opcional de restricciones físicas para mejorar la transportabilidad, ofreciendo una base reproducible para investigación en predicción de disrupciones y un camino práctico hacia la integración en tiempo real en dispositivos actuales y, con adaptación de dominio, en máquinas de mayor tamaño como \ac{ITER}.
