Disruptions in tokamak plasmas—rapid losses of thermal energy followed by current quench driven by \ac{MHD} instabilities—pose severe thermo-mechanical and electromagnetic risks that scale unfavorably with device size. Anticipating these events with sufficient lead time is a prerequisite for machine protection and for enabling effective termination or mitigation (e.g., massive gas or pellet injection). 
This work presents a protocol-driven ensemble for disruption assessment that integrates heterogeneous models behind a unified \acs{HTTP}/OpenAPI interface, enabling composability, versioning, and deployment across languages and runtimes. 
The data model mirrors the experimental setup: seven synchronous diagnostics sampled every 1\,ms from shot start until termination or disruption. Crucially, class imbalance arises primarily from discharge prevalence (\(\sim\!6\%\) of shots are disruptive); the additional length disparity between non-disruptive (\(\sim\!12\mathrm{k}\) samples) and disruptive (\(\sim\!7\mathrm{k}\) samples) shots further amplifies imbalance when training on fixed-size windows. 
Preprocessing is causal in every prediction model: windowed features and normalizers operate without access to future samples.
Physics-informed descriptors—mean values per window and DC-free spectral content, complemented by inter-signal ratios such as radiated-to-input power fraction, Greenwald fraction, and inductance/current normalizations—capture relevant precursors while preserving interpretability. 
The ensemble supports both fixed-length evaluation and rolling early-warning on streams, exposing calibrated scores and justifications at the window level to facilitate conservative decision logic in the orchestrator (thresholding and refractory policies). 
Generalization is treated as a first-class concern: the literature highlights the poor extrapolation of device-specific predictors and the need for well-normalized multi-machine, dimensionless datasets for next-step applicability. 
Accordingly, the workflow centers on cross-campaign validation, diagnostics of covariate shift, and optional physics constraints to improve transportability. 
The resulting toolkit delivers a reproducible baseline for disruption prediction research and a practical path toward real-time integration, while explicitly acknowledging the limits of training on historical data from a single tokamak and the consequent need for domain adaptation before deployment on larger devices such as \ac{ITER}.
