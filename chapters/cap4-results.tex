\chapter{Results}\label{sec:cap4}

This chapter presents the results obtained from the implemented models, including their performance metrics and comparisons. The models are evaluated based on their ability to detect anomalies in the discharge data and the lead time they provide before a disruption occurs.

\section{GitHub Organization}

This project uses git to handle different versions. As multiple software has been developed in this project, a GitHub Organization has been created in order to ease the cloning process.

The main repositories available on the organization are:

\begin{itemize}
    \item \textbf{Outlier orchestrator:} This repository contains the orchestrator explained at \autoref{sec:orchestrator}. 
    \item \textbf{py\_xgboost:} This repository contains the implementation of the XGBoost model \autocite{OutlierClassifierPy_xgboost2025}.
\end{itemize}

\section{Model Analysis and optimization}

This section provides an analysis of the models implemented in this project, focusing on their performance metrics and the optimization techniques applied to enhance their accuracy and efficiency. To do so, 12 discharges from the C24 campaign are selected for a deep analysis. These discharges are representative of different operational scenarios and include both normal and anomalous events. 

\subsection{XGBoost}

XGBoost implementation uses the features described in \autoref{subsubsec:xgboost}, which includes features from the current window and past windows as a tendency. This approach allows the model to have a better understanding of the discharge behavior over time, but also makes the model more rigid to changes in the discharge pattern.


